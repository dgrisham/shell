\documentclass[]{article}
\usepackage{lmodern}
\usepackage{amssymb,amsmath}
\usepackage{ifxetex,ifluatex}
\usepackage{fixltx2e} % provides \textsubscript
\ifnum 0\ifxetex 1\fi\ifluatex 1\fi=0 % if pdftex
  \usepackage[T1]{fontenc}
  \usepackage[utf8]{inputenc}
\else % if luatex or xelatex
  \ifxetex
    \usepackage{mathspec}
    \usepackage{xltxtra,xunicode}
  \else
    \usepackage{fontspec}
  \fi
  \defaultfontfeatures{Mapping=tex-text,Scale=MatchLowercase}
  \newcommand{\euro}{€}
\fi
% use upquote if available, for straight quotes in verbatim environments
\IfFileExists{upquote.sty}{\usepackage{upquote}}{}
% use microtype if available
\IfFileExists{microtype.sty}{%
\usepackage{microtype}
\UseMicrotypeSet[protrusion]{basicmath} % disable protrusion for tt fonts
}{}
\ifxetex
  \usepackage[setpagesize=false, % page size defined by xetex
              unicode=false, % unicode breaks when used with xetex
              xetex]{hyperref}
\else
  \usepackage[unicode=true]{hyperref}
\fi
\usepackage[usenames,dvipsnames]{color}
\hypersetup{breaklinks=true,
            bookmarks=true,
            pdfauthor={David Grisham},
            pdftitle={Project 2 -- Shell},
            colorlinks=true,
            citecolor=blue,
            urlcolor=blue,
            linkcolor=magenta,
            pdfborder={0 0 0}}
\urlstyle{same}  % don't use monospace font for urls
\setlength{\parindent}{0pt}
\setlength{\parskip}{6pt plus 2pt minus 1pt}
\setlength{\emergencystretch}{3em}  % prevent overfull lines
\providecommand{\tightlist}{%
  \setlength{\itemsep}{0pt}\setlength{\parskip}{0pt}}
\setcounter{secnumdepth}{0}

\title{Project 2 -- Shell}
\author{David Grisham}
\date{30 September, 2015}

% Redefines (sub)paragraphs to behave more like sections
\ifx\paragraph\undefined\else
\let\oldparagraph\paragraph
\renewcommand{\paragraph}[1]{\oldparagraph{#1}\mbox{}}
\fi
\ifx\subparagraph\undefined\else
\let\oldsubparagraph\subparagraph
\renewcommand{\subparagraph}[1]{\oldsubparagraph{#1}\mbox{}}
\fi

\begin{document}
\maketitle

\subsubsection{Usage}\label{usage}

Run \texttt{make} in the \texttt{src/} directory to build the project.
If all goes well, then you should be left with an executable called
\texttt{myshell} in the \texttt{src/} directory. Running this will open
the shell.

\subsubsection{List of Files}\label{list-of-files}

\begin{itemize}
\tightlist
\item
  \texttt{shell.cpp}

  \begin{itemize}
  \tightlist
  \item
    main shell process that sets up the environment, then (basically)
    follows these steps:

    \begin{enumerate}
    \def\labelenumi{\arabic{enumi}.}
    \tightlist
    \item
      display prompt to user
    \item
      read in user input (currently only accepts built-in commands)
    \item
      tokenize input and handle the following

      \begin{itemize}
      \tightlist
      \item
        local variable assignment
      \item
        variable substitution
      \item
        alias substitution
      \end{itemize}
    \item
      execute the function corresponding to the user input
    \end{enumerate}
  \end{itemize}
\item
  \texttt{builtins.cpp}

  \begin{itemize}
  \tightlist
  \item
    contains an implementation for each command that's built-in to the
    shell. see the `Builtins' section for more details
  \end{itemize}
\item
  \texttt{builtins.h}

  \begin{itemize}
  \tightlist
  \item
    header file for the functions defined in \texttt{builtins.cpp}
  \end{itemize}
\end{itemize}

\subsubsection{Builtins}\label{builtins}

The following builtin functions are currently supported by this shell
(Note: terms surround by \texttt{\textless{}\textgreater{}} (e.g.
\texttt{\textless{}term\textgreater{}}) should be replaced by user
input.):

\begin{itemize}
\tightlist
\item
  \texttt{ls}, \texttt{ls\ \textless{}directory\textgreater{}}

  \begin{itemize}
  \tightlist
  \item
    with no argument, lists the files in the current working directory
  \item
    otherwise, lists the files in
    \texttt{\textless{}directory\textgreater{}}
  \end{itemize}
\item
  \texttt{cd\ \textless{}directory\textgreater{}}

  \begin{itemize}
  \tightlist
  \item
    if \texttt{cd} is called with no arguments, the shell will change
    into the directory defined by \$HOME (usually
    \texttt{/home/\textless{}username\textgreater{}})
  \item
    otherwise, changes the current working directory to
    \texttt{\textless{}directory\textgreater{}}
  \end{itemize}
\item
  \texttt{echo\ \textless{}arg\textgreater{}}

  \begin{itemize}
  \tightlist
  \item
    prints \texttt{\textless{}arg\textgreater{}} to stdout, followed by
    a newline
  \end{itemize}
\item
  \texttt{pwd} displays the value held in the \texttt{\$PWD} environment
  variable. It is left up to other functions (like \texttt{cd}) to
  update \texttt{\$PWD} when necessary
\item
  \texttt{alias},
  \texttt{alias\ \textless{}cmd\textgreater{}=\textless{}def\textgreater{}}

  \begin{itemize}
  \tightlist
  \item
    with no arguments, \texttt{alias} will show all of the currenty
    defined aliases
  \item
    otherwise, \texttt{\textless{}cmd\textgreater{}} is the alias name,
    and \texttt{\textless{}def\textgreater{}} is the command that should
    be called when \texttt{\textless{}cmd\textgreater{}} is called
  \end{itemize}
\item
  \texttt{unalias\ -a}, \texttt{unalias\ \textless{}cmd\textgreater{}}

  \begin{itemize}
  \tightlist
  \item
    with the \texttt{-a} option, unalias will remove all currently
    defined aliases
  \item
    \texttt{\textless{}cmd\textgreater{}} is the name of the alias that
    should be removed from the aliases list
  \end{itemize}
\item
  \texttt{exit}

  \begin{itemize}
  \tightlist
  \item
    exists the shell with an exit code EXIT\_SUCCESS
  \end{itemize}
\item
  \texttt{history}, \texttt{history\ \textless{}N\textgreater{}}

  \begin{itemize}
  \tightlist
  \item
    with no arguments, \texttt{history} displays the entire history of
    shell commands ran in the current session
  \item
    \texttt{\textless{}N\textgreater{}} is the number of commands in the
    history to display, starting with the most recent
  \end{itemize}
\item
  \texttt{!!}

  \begin{itemize}
  \tightlist
  \item
    executes the last command that was ran
  \end{itemize}
\item
  \texttt{!\textless{}N\textgreater{}}

  \begin{itemize}
  \tightlist
  \item
    executes the \texttt{N}th command in the history
  \end{itemize}
\item
  \texttt{!-\textless{}N\textgreater{}}

  \begin{itemize}
  \tightlist
  \item
    executes the \texttt{N}th-preceding command (same as Bash)
  \end{itemize}
\end{itemize}

\subsubsection{Unusual / Interesting
Features}\label{unusual-interesting-features}

As noted in the Builtins section:

\begin{itemize}
\tightlist
\item
  \texttt{cd} defaults to the home directory when called with no
  arguments
\item
  \texttt{history\ \textless{}N\textgreater{}} can be used to see the
  \texttt{N} most recent lines in the history
\item
  \texttt{!-\textless{}N\textgreater{}} can be used to execute the
  \texttt{N}th-preceding line in the history (\texttt{N} is the number
  of the desired line (in the history) relative to the current line)
\end{itemize}

As point of potential interest, the way in which aliases are handled
allows the user to define nested aliases without issue. For example:

\begin{verbatim}
prompt > alias e=echo
prompt > e hello
hello 
prompt > alias a=e
prompt > a hello
hello 
\end{verbatim}

\subsubsection{Number of Hours Spent}\label{number-of-hours-spent}

Intermediate 1 -- Around 6 hours

\end{document}
